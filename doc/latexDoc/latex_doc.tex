\documentclass{article}
% Language setting
% Replace `english' with e.g. `spanish' to change the document language
\usepackage[english]{babel}

% Set page size and margins
% Replace `letterpaper' with`a4paper' for UK/EU standard size
\usepackage[letterpaper,top=2cm,bottom=2cm,left=3cm,right=3cm,marginparwidth=1.75cm]{geometry}

% Useful packages
\usepackage{amsmath}
\usepackage{amssymb}
\usepackage{graphicx}
\usepackage[colorlinks=true, allcolors=blue]{hyperref}

\title{Tolosat's Gravimetry Payload Report}
\author{}

\begin{document}
\maketitle

\begin{abstract}
Tolosat's gravimetry payload aims to a build a spherical harmonic model of the Earth's gravitational potential.

\end{abstract}

\section{Introduction}
Accurate knowledge of the gravitational potential of the Earth, 
on a global scale and at very high resolution, is a fundamental prerequisite 
for various geodetic, geophysical and oceanographic investigations and applications. 
Over the past 50 or so years, continuing improvements and refinements to the 
basic gravitational modeling theory have been paralleled by the availability 
of more accurate and complete data and by dramatic improvements in the 
computational resources available for numerical modeling studies. 

These advances have brought the state-of-the-art from the early spherical harmonic models of degree 8 
\cite{1952Tegf}

\section{Method for unnoised model}

The Earth's external gravitational potential, V, 
at a point P defined by its geocentric distance (r), 
geocentric co-latitude ($\theta$) (defined as 90°-latitude), 
and longitude ($\lambda$), is given by:
\begin{equation}
        V (r, \theta, \lambda) = \frac{GM}{r}  \left[1 + \sum_{n = 2}^{\infty} \left(\frac{R}{r}\right)^{n} \sum_{m = 0}^{n} P_{nm}(cos(\theta)) \left[C_{nm}cos(m\lambda) + S_{nm} sin(m\lambda)\right]    \right] 
\end{equation}

where
\begin{equation*}
    \left\{
                \begin{array}{ll}
                R = \text{the Earth’s equatorial radius} \\
                \\
                P_{nm} = \text{the fully normalized associated Legendre
    functions of degree n and order m}\\
    \\
                C_{nm} S_{nm} \text{the geopotential harmonic coefficients. }
                \end{array}
    \right .
\end{equation*}


We rewrite the geopotentiel harmonic series  (1) :

\begin{equation}
    V(r) = \sum_{n,m } \left[ C_{nm} V_{nm}^{c}(r) + S_{nm} V_{nm}^{s}(r) \right]  
\end{equation}

This form shows explicitly the linearity of teh harmonic coefficients $C_{nm}$ et $S_{nm}$ \\


The acceleration approach is based on the Newton's second law of motion, which links the acceleration vector to the gradient of the gravitational potential.

we obtain : 
\[
\frac{d^2 r}{d t^2} = a_{grav} + a_{\text{lunisolar perturbations}} + a_{\text{solid Earth and ocean tides}} + a_{\text{correction due to general relativity}} + a_{\text{non gravitational acceleration}}
\]

\subsection{A hypothesis on the accelerations}

We studied the weigth and importance of the accelarations. We conclude: 
\begin{equation*}
    a_{grav}  >>> a_{others} 
\end{equation*}
So,for the rest of the work we will concider that any accelerations other than $a_grav$ is negligeable.

\subsection{A numerical approximation of the second derivative of the position}
\begin{equation}
    \frac{d^2 r}{dt^2} \approx \frac{d^2 Q(r_{gps})}{dt^2} = F * r_{gps} 
\end{equation}

Where $r_{gps}$ is the GPS position. We do this approximation using a polynomial smoothing filter.

We use as a filter the Savitzky-Golay filter 

% idea We can elaborate the idea of SG filter


\vspace{1 cm}

We obtain the observation equation

\begin{equation}
    \frac{d^2 Q (r_{gps})}{dt^2} - a_{other} = \sum_{n,m }   \left[ C_{nm} \bigtriangledown V_{nm}^{c}(r) + S_{nm} \bigtriangledown V_{nm}^{s}(r) \right]  
\end{equation}
By solving this system, we obtain the harmonic coefficients. 

We wrtie the model:
\begin{equation}
    y = A x    
\end{equation}

the vector of observations $y = \frac{d^2 Q(r_{gps})}{dt^2}$\\

the design matrix $A = \left[ \nabla V_{nm}^c ; \nabla V_{nm}^s \right]  $\\

the estimated parameters $x= \left[C_{nm} ; S_{nm}\right]$ 

Intuitively, one can think about solving the problem with: 

\begin{equation*}
    x = (A^TA)^{-1} A^T y
\end{equation*}

However, the dimensions of the matrix A are big. Numerical calculus would be very sensible to numerical errors. 

We use a classic least square method to solve this problem. 
\subsection{Resultats}
Our first experiences showed that this model is very sensible towards small variations of $\theta ,  \phi$.

We suggest to use a projection on a space. 

Let this projection be: 

\begin{equation}
    \begin{array}{l|rcl}
        P_E : & \mathbb{R}^3 & \longrightarrow &\mathbb{R}^3   \\
            
        & \left(\begin{array}{ll}
                r \\
                \theta \\
                \phi
                \end{array}
            \right )   & \longmapsto & \left(
                \begin{array}{ll}
                r \\
                0 \\
                0
                \end{array}
            \right ) \end{array}
\end{equation}

In other words, we focus on the radial acceleration. 

Let \begin{equation*}
    A x = y \Leftrightarrow P_E A x =P_E y = \tilde{y}  
    \implies x \approx  \tilde{x} = Px
\end{equation*}


\section{Method for noised model }

Let us suppose, for instance, that the measurements of the positions are noised. 

\textit{i.e} 



\begin{align*}
    p &= \tilde{p} + \epsilon \\ 
    Sp &= S\tilde{p} + S\epsilon \text{\hspace{1 cm}where S is the Savitzky Golay Filter }\\
    x &= \tilde{x} + S\epsilon\\
    x &= (A^TA)^{-1} A^T  S \tilde{p} +  (A^TA)^{-1} A^T  S \epsilon 
\end{align*}

If we suppose that $ \epsilon \sim \mathcal{N}(0, \sigma I) $ we can obtain , 

\[   \epsilon_2 = Q \epsilon   \text{\hspace*{1 cm}} \mathbb{E}[\epsilon \epsilon^T]  = \sigma^2 Q Q^T  
\]



\textbf{Note that we should add in the equation a matrix B of noise decorrelation  }







\newpage
\bibliographystyle{plain}
\bibliography{biblio}


\end{document}
